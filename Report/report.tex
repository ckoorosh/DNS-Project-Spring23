\documentclass{article}

\usepackage{graphicx}
\usepackage{indentfirst}
\usepackage{booktabs}
\usepackage[a4paper, total={6in, 8in}]{geometry}
\usepackage{hyperref}
\usepackage{fancyhdr}
\usepackage{subcaption}
\usepackage{amsmath}
\usepackage{xepersian}
\usepackage{fontspec}
\usepackage{url}
\settextfont[Scale=1,ExternalLocation=fonts/,BoldFont=B Nazanin Bold.ttf]{B Nazanin}
\setlatintextfont[Scale=1,ExternalLocation=fonts/,BoldFont=XB Zar.ttf]{Times New Roman}

\begin{document}


%title page%
\begin{titlepage}
	\begin{center}
		\vspace{0.2cm}
		
		\includegraphics[width=0.25\textwidth]{sharif.png}\\
		\vspace{0.2cm}
		
		\Large{پروژه امنیت داده و شبکه}\\
		\vspace{1.0cm}
		
		\textbf{ \Huge{پیام‌رسان امن}}\\
		\vspace{1.0cm}

		\textbf{ \Large{استاد}}\\
		\textbf{ \Large{دکتر مرتضی امینی}}\\
		\vspace{0.5cm}
		
		\textbf{\Large{نویسندگان}}\\
		\textbf{\Large{محمدرضا مفیضی - 98106059}}\\
		\textbf{\Large{رضا علیپور - 98105932}}\\
		\textbf{\Large{محمدعلی کاکاوند - 98102119}}\\
		\vspace{1.0cm}
		
		\large   دانشگاه صنعتی شریف\\\vspace{0.1cm}
		\large دانشکده مهندسی کامپیوتر\\\vspace{0.2cm}
		\large   بهار ۱۴۰۲ \\\vspace{0.1cm}
	\end{center}
\end{titlepage}
%title page%

\newpage
\tableofcontents
\newpage
%pages header
\pagestyle{fancy}
\fancyhf{}
\fancyfoot{}
\setlength{\headheight}{59pt}
\cfoot{\thepage}
\lhead{پیام‌رسان امن}
\rhead{\includegraphics[width=0.1\textwidth]{sharif.png}\\
		دانشکده مهندسی کامپیوتر
}
\chead{پروژه امنیت داده و شبکه}
%pages header


\section{مقدمه}
در این نوشته، سعی می‌کنیم توضیحاتی درباره نحوه پیاده‌سازی پیام‌رسان امن با ویژگی‌های خواسته‌شده در سند پروژه ارائه دهیم.
این پیام‌رسان از رمزنگاری انتها به انتها برای تبادل پیام استفاده می‌کند و به صورت کلاینت‌-سروری پیاده‌سازی شده است.
پیام‌های رد و بدل شده میان هر کلاینت و سرور نیز با استفاده از کلید مشترک بین آن‌ها رمز خواهد شد.

برای پیاده‌سازی بخش سرور از چارچوب \lr{Django} استفاده شده است و درخواست‌‌ها از طریق پروتکل \lr{Rest} به سرور ارسال می‌شود.
برای ارسال پیام از سرور به کلاینت نیز از پروتکل \lr{WebSocket} استفاده می‌شود.
جزئیات رمزگذاری انتها به انتها و پیاده‌سازی سرور و کلاینت در ادامه آمده است.

\section{کلاینت}
\subsection{کلیات}
در سمت کلاینت جفت‌کلید خصوصی و عمومی او برای رمزنگاری نامتقارن جهت ایجاد نشست با سرور و یا کلاینت دیگر نگه‌داری می‌شود.
برای ارسال پیام به سرور فقط از \lr{post} استفاده می‌شود و همواره توکنی که سرور برای نشست به ما داده است در کنار پیام قرار گرفته و همگی با کلید نشست با سرور رمز می‌شوند.

\subsection{ایجاد حساب کاربری}
برای ایجاد و وارد شدن به حساب کاربری، کلاینت باید نام کاربری و رمز عبور را از او دریافت کند.
سپس این اطلاعات برای ثبت‌نام به سرور فرستاده می‌شود.
همچنین در صورتی که ثبت‌نام با موفقیت انجام شد، کاربر کلید عمومی خود به همراه پیش‌کلید امضاشده و امضای آن و یک لیست از پیش‌کلیدهای یک‌بار مصرف را برای سرور ارسال می‌کند.

ورود به حساب کاربری نیز با گرفتن نام کاربری و رمز عبور انجام می‌شود.
در پاسخ یک توکن به کاربر داده می‌شود تا در درخواست‌های بعدی از آن استفاده شود.

\subsection{ارسال و دریافت پیام}
برای ارسال پیام به یک کاربر و رمزگذاری انتها به انتهای آن به این صورت عمل می‌کنیم.

برای ارسال پیام از پروتکل \lr{Double Ratchet} استفاده می‌کنیم.
این پروتکل ویژگی‌های محرمانگی پیش‌رو و پس‌رو را برای ما فراهم می‌کند.
جزئیات بیشتر این پروتکل در \href{https://signal.org/docs/specifications/doubleratchet/}{این لینک} آمده است.

در صورتی که کاربر دیگر آفلاین باشد از الگوریتم \lr{3DH} استفاده می‌شود.
به این صورت که ابتدا ۳ کلید \texttt{identity key}، \texttt{signed prekey} و \texttt{prekey signature} کاربر آفلاین را (که در هنگام ثبت‌نام با سرور به اشتراک گذاشته است) از سرور دریافت می‌کنیم.
سپس سه بار با کلیدهای دریافت‌شده \lr{DH} می‌زنیم:
\begin{align*}
	&DH1 = DH(IK_A, SPK_B) \\
	&DH2 = DH(EK_A, IK_B) \\
	&DH3 = DH(EK_A, SPK_B) \\
	&DH4 = DH(EKA, OPKB) \\
	&SK = KDF(DH1 || DH2 || DH3 || DH4)
\end{align*}
درنهایت به یک کلید خواهیم رسید که پیام را با استفاده از آن رمز و به سرور ارسال می‌کنیم.
سرور نیز هنگام آنلاین‌شدن فرد مقابل پیام رمز‌شده را به او ارسال می‌کند.

به دلیل استفاده‌شدن از \texttt{one-time prekey} امکان اجرای حمله تکرار نخواهد بود.
همچنین به دلیل استفاده از امضا روی پیام‌های امکان انکار وجود نخواهد داشت.

جزئیات بیشتر این روش در \href{https://signal.org/docs/specifications/x3dh/#x3dh-parameters}{این لینک} قابل مشاهده است.


\subsection{نگه‌داری پیام‌ها به‌صورت امن}
پیام‌های هر کاربر با کاربران دیگر توسط کلیدی که توسط کلید مشتق‌شده از رمز عبور فرد ساخته می‌شود رمز می‌شوند.
با ورود کاربر به حساب کاربری خود می‌توان پیام‌های قبلی را مشاهده کرد.

\subsection{نگه‌داری کلید به‌صورت امن}
کلیدهای هر کاربر با کاربران دیگر نیز با استفاده از کلید گفته‌شده در بخش قبلی رمز می‌شوند.

\subsection{ایجاد و مدیریت گروه}
برای ایجاد یک گروه کاربر درخواست خود را به همراه نام گروه به سرور ارسال می‌کند.

برای ساخت کلید گروه از الگوریتم \lr{AES} برای ساخت یک کلید نشست متقارن استفاده می‌شود.

کاربر به صورت پیش‌فرض ادمین گروه خواهد بود و می‌تواند کاربران آنلاین دیگر را با ارسال نام کاربری آن‌ها به سرور به گروه اضافه کند.
بعد از اضافه‌شدن کاربر لازم است تا کلید نشست گروه با او به اشتراک گذاشته شود.

برای تبادل کلید با کاربر جدید ابتدا با استفاده از پروتکل \lr{DH}  یک کلید موقت برای انتقال کلید استفاده می‌شود.
سپس کلید نشست اصلی از طریق این کلید یک‌بار مصرف با کاربر جدید داده می‌شود.

\subsection{تایید صحت نشست}
برای تایید صحت نشست کاربران می‌توانند با انتخاب گزینه مورد نظر ، چکیده بخشی از کلید نشست را به صورت شکلک‌هایی مشاهده کنند و با مقایسه آن در دو سمت اتصال از امن بودن ارتباط اطمینان حاصل کنند.

\subsection{تازه‌سازی کلیدهای نشست}
هر کاربر می‌تواند با انتخاب گزینه‌ای کلید نشست جدیدی را برای ارتباط تولید کند و بقیه پیام‌ها با کلید جدید رمز خواهند شد.
هنگام حذف یک کاربر از گروه نیز همین اتفاق خواهد افتاد و با تبادل کلید جدید نشست بروزرسانی می‌شود.

\section{سرور}
\subsection{کلیات}
سرور فقط درخواست‌های \lr{post} را می‌پذیرد و برای ارتباط با کاربران از کلید نشست تبادل‌شده استفاده می‌کند.
همچنین سرور دسترسی به پیام‌های انتها به انتهای دو کاربر که با کلید نشست بین آن دو رمز شده است ندارد.
برای هر درخواست از سمت کلاینت نیز، سرور توکن داده‌شده را بررسی می‌کند و در صورت صحت درخواست را انجام می‌دهد.

\subsection{ایجاد حساب کاربری}
در سمت سرور به هنگام دریافت نام کاربری و رمز عبور، یک \lr{salt} به صورت تصادفی ایجاد می‌شود و چکیده رمز با \lr{salt} در پایگاه‌داده ذخیره می‌شود.
با این کار حتی در صورت حمله به پایگاه‌داده نیز رمزهای کاربران لو نمی‌رود.
هنگام لاگین نیز سرور رمز داده‌شده را کنار \lr{salt} ذخیره‌شده می‌گذارد و با مقدار ذخیره‌شده در پایگاه داده مقایسه می‌کند.

\subsection{نمایش کاربران آنلاین}
برای نمایش کاربران آنلاین کاربرهایی را که اتصال \lr{WebSocket} فعال دارند به کاربر ارسال می‌کنیم.

\subsection{ایجاد و مدیریت گروه}
ایجاد گروه با درخواست کاربر به سرور انجام می‌شود.
سرور اطلاعات گروه‌‌ها به همراه کاربران و نقش‌ها آن‌ها را ذخیره می‌کند.

\subsection{ارسال پیام‌ها به مقصد}
سرور با دریافت پیام از سمت یک کاربر و نام کاربری فرد مقصد آن را با استفاده از اتصال \lr{WebSocket} به مقصد ارسال می‌کند.
سرور پیام دریافت‌شده را در صورتی که کاربر دیگر آنلاین باشد در سمت خود ذخیره نخواهد کرد.

\section{نیازمندی‌های امنیتی}
برقراری نیازمندی‌های امنیتی خواسته‌شده را به‌صورت خلاصه در زیر شرح می‌دهیم:
\begin{enumerate}
	\item 
	پیام‌های رد و بدل شده بین کاربران همگی با استفاده از پروتکل‌های توضیح داده‌شده به صورت انتها به انتها رمز می‌شوند.
	\item
	در زمان توافق کلید بین دو کاربر، به‌دلیل استفاده از نانس و برچسب زمانی ویژگی تازگی برقرار خواهد بود.
	از طرفی خودِ پروتکل \lr{Double Ratchet} حافظ ویژگی تازگی برای ما خواهد بود زیرا برای هر پیام از یک رمز جدید استفاده می‌شود.
	\item
	هر پیامی که بین دو کاربر و بین کاربر و سرور ارسال می‌شود توسط کلید خصوصی رمز می‌شود تا صحت و یکپارچگی پیام‌ها حفظ شود.
	\item
	به دلیل استفاده از برچسب زمانی برای ارسال پیام‌ها سازگاری آن‌‌ها هنگام ارسال برقرار خواهد بود.
	\item 
	برای ویژگی‌های احراز اصالت و عدم انکار تمامی پیام‌ها توسط کلید نامتقارن امضا می‌شوند و امضا به همراه پیام ارسال می‌شود.
	\item
	امکان اضافه‌کردن عضو جدید به گروه تنها توسط ادمین‌ها قابل انجام است و در سمت سرور این درخواست فقط از کاربران ادمین پذیرفته می‌شود.
	\item 
	حمله مرد میانی نیز به دلیل رمزگذاری انتها به انتها بین هر دو کاربر و همچنین رمزگذاری بین کلاینت و سرور امکان‌پذیر نخواهد بود.
	\item 
	حمله تکرار نیز مشابها به دلیل استفاده از الگوریتم \lr{Double Ratchet} و برچسب زمانی امکان‌پذیر نخواهد بود.
	\item
	الگوریتم‌های رمزگذاری استفاده شده همگی از روش‌های بروز و مورد تایید هستند و در حال حاضر امن می‌باشند.
	\item 
	ویژگی محرمانگی پیش‌رو به دلیل استفاده از پروتکل \lr{Double Ratchet} برقرار می‌باشد.
	زیرا در صورت لو رفتن کلید بلندمدت کلیدهای نشست قبلی قابل بازیابی نخواهند بود و محرمانگی پیام‌هایی که درگذشته تبادل شده‌اند حفظ می‌شود.
	\item 
	محرمانگی پس‌رو نیز به‌دلیل استفاده از پروتکل \lr{Double Ratchet} و استفاده از \lr{chain}ها برقرار خواهد بود.
	
\end{enumerate}


\end{document}
